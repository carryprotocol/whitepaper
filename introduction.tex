\section{Introduction}
Carry introduces an innovative advertising protocol tailored for the gaming industry, offering the potential for increased revenue for game developers and players. By replacing traditional advertising platforms, this protocol not only enhances the advertising effectiveness for advertisers but also reduces associated costs.

The Carry Protocol innovatively bridges the worlds of gaming and advertising within the web3 space, offering a unique system for showcasing ads within virtual game environments. These advertising spaces, or "slots," are thoughtfully integrated into games, functioning like digital billboards for interactive promotions. This approach aims to create a better and healthier gaming advertising economics.

Moreover, the protocol incorporates a transparent and equitable auction system to allocate advertising space, where the market value of each slot is determined by real-time demand. This system ensures a fair and open process for all participants.

Carry Protocol aims to challenge traditional advertising approaches within the gaming industry by seamlessly integrating advertisements, benefiting game developers, players, and advertisers alike.


\subsection{Problems with Game Advertisement Market}
\begin{itemize}[leftmargin=*]
    \item \textbf{Failed Incentives:} The dominance of the play-to-earn dual-token economic model in many web3 games, although initially promising, has demonstrated its limitations. Due to its aggressive and singular incentive structure, it often places developers and players in positions of discordant interests, threatening the game’s longevity. In the later stages of the vast majority of play-to-earn games, due to the singularity of incentive mechanisms, players can solely profit by selling their held NFT assets. Regardless of the developers’ efforts to salvage or enhance the gaming experience, games remain highly susceptible to entering a "death spiral" within a relatively short timeframe. 
    
    For a sustainable future, web3 games necessitate diversified economic mechanisms that foster collaborative wins among developers, players, and other stakeholders.
    \item \textbf{Platform Monopoly:} Developers have grown weary of the "Apple tax", which, as a giant advertisement platform, dominates the web2 advertising system and web2 monetization standards by monopolizing user data assets. The web2 advertising ecosystem primarily comprises four key players: advertisers, media, advertising platforms, and users. Advertising platforms, by monopolizing user data assets, shape the advertising monetization landscape. However, the advancement of privacy technology and blockchain technology has provided a mature technological solution for a fairer advertising protocol.
 
    Both developers and users have a strong demand for a more equitable and efficient advertising ecosystem. The current dominance of giant advertising platforms has significantly impacted business and innovation efficiency.

    \item \textbf{Inability to Analyze Player Intent:} Despite the transformative potential of web3 in gaming, many developers operate based on speculative insights about player preferences, leading to games that either wane in popularity or serve primarily for asset arbitrage. The heart of this challenge is the absence of effective tools to analyze user behavior within Web3 games. This oversight hinders developers from truly understanding player intent. Notably, games like Crypto Kitties, which introduced users to the creation and trade of in-game NFTs, underscore the vast possibilities but also the need for more nuanced player behavior insights.

    Using the crypto-native game ’Loot’ as a case study, it’s evident that following its significant traction within the NFT community, over 20 development teams have emerged with a focus on Loot. These teams are diligently working on an array of games rooted in the foundational model of Loot. However, a minimal number of these endeavors have shown tangible advancement. To solve these issues, the developers and players need to have more interaction during its life lifecycle. Native features in web3, for example, predict the market, provide methods for developers to truly interact with players.
 
    As a slogan in the web3 industry goes, "onchain is the new online." Therefore, a multitude of data dimensions can now be easily captured. We need greater insights into users to enhance the efficiency of advertising and gaming.
\end{itemize}

\subsection{Slot Dynamics}
At the heart of the Carry Protocol lies the innovative concept of a "slot," a transformative idea designed to bridge the worlds of advertising and on-chain gaming. In its essence, a slot is a unique digital real estate where advertisements can be strategically placed, integrated seamlessly into the virtual environments of games or other digital platforms.

\begin{figure}[h]
    \centering
    \includegraphics[width=0.6\textwidth]{slot.jpg}
    \caption{A Slot by Design}
    \label{fig:slot_basic}
\end{figure}

\begin{itemize}
    \item Every slot is inherently tied to a designated game, taking forms such as visual spaces for graphics, character-specific sound effects, or virtual companions roaming the digital environment.
    \item Beyond their in-game presence, slots hold asset value and can be traded or transferred among participants.
    \item Slots can be a carrier of various types of media, videos, pictures etc.
\end{itemize}

Imagine you're playing an online game, and as you explore, you see ads on billboards or even on items like a character's backpack. These are what we call "slots" in the Carry Protocol. Think of a slot like a piece of virtual real estate where ads can live. These slots are valuable to advertisers and can be bought, sold, or traded just like other virtual items in the game.

Slots are versatile. They can show video ads, become part of the game scenery, or even be part of the outfit your character is wearing. They're designed to fit into the game naturally, so they're interesting rather than annoying.

The worth of these ad spaces can go up or down. For example, a slot right where players hang out the most could be worth a lot because more players will see the ad there, while a slot off in a quiet corner might not be as valuable.

The Carry Protocol is all about changing up the game when it comes to ads. These slots are active parts of the game that can change based on how players interact with them and what advertisers want to try. They're a new way to think about ads in games, making sure that players' experiences come first.